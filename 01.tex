\documentclass[UTF8,fontset=fandol]{ctexart}
\title{猫站小报 第 1 期}
\author{猫站小报编辑部}
\date{\today}
\usepackage{indentfirst}
\setlength{\parindent}{2em}
\usepackage{listing}
\usepackage{xcolor}

\begin{document}
\maketitle
\part{编辑部专版}
\section{发刊词}

\noindent
\textbf{各位Kitten作者,Python诗人:} 

编程猫社区一直以来是国内最大的少儿编程社区。2023年以来,官方的不作为使得社区比较为所,社区长期缺乏正经的技术性内容,而我们则选择一种偏传统的方式——报刊。来为编程技术留下一些思考的绿地

\noindent
\textbf{我们的小报分为四部分:}

\begin{itemize}
\item 编辑部专版 - 这里会刊登一些科技界的新闻,社区的时间,编辑部的社论
\item 积木纪元 - 这里会刊登一些优秀的Kitten作品,被官方忽视的优秀作品我们会为你刊登
\item 代码诗篇 - 刊登一些优秀的代码编程作品,官方并不重视Python作品,你可以在这里找到那些被忽视的代码
\item 传火者 - 刊登一些教程类的文章,为了提升全社区的编程水平
\end{itemize}

除了第一部分外,所有部分均接受投稿,编辑们的联系方式可以在各版页脚找到。我们计划每周六在社区出刊,当然,作为一个社区项目,八分靠热情,长时间不出刊也有可能发生,你在我们在社区的帖子之下的跟帖就是对我们最大的支持。

\begin{verbatim}
	print("Hello Codemao Newsletter)
\end{verbatim}

\part{积木纪元}

\part{代码诗篇}

\part{传火者}
\end{document}